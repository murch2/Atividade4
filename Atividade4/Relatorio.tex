%Tudo que começa com '%' é comentário e é ignorado pelo compilador

%Gerando arquivo em latex:
%latex arquivo.tex (em dvi)
%pdflatex arquivo (em pdf)
%dvipdfm arquivo
%s2pdf arquivo

% Alguns modos de usar o latex:
% Windows – Miktex com Led
% Linux – texlive com kile

\documentclass[12pt]{article} %aqui fala o tipo de documento e o tamanho da fonte. Opções: tamanho do texto (10pt, 12pt, 14pt), formato do papel (a4paper, a5paper, b5paper, letterpaper, legalpaper, executivepaper), o número de colunas (onecolumn, twocolumn), entre outras opções.
%Por exemplo, [12pt,a4,twocolumn].
%classe: article, report, letter, book ou slides. Instalar abnt para quem está pensando no tf
\usepackage[T1]{fontenc}
\usepackage[utf8]{inputenc}
\usepackage{lmodern}
\usepackage[brazil]{babel}
\usepackage[labelformat=empty]{caption}


\begin{document} % Aqui começa o documento
\title{Relatório - Exercício 4: Implementação dos Requisitos de Teste} % título
\author{Rodrigo Duarte Louro -- António Martins Miranda} % quem escreveu
\date{06/05/2014} %data

\maketitle %cria o título

\newpage

\section{Utilização do Recoder}
\mbox{} %As vezes, para o primeiro parágrafo da seção sair com espaçamento correto precisa dessa gambiarrinha

	A ferramenta recoder foi utilizada em praticamente todos os passos do desenvolvimento desse projeto. 
A Árvore de análise sintática gerada por essa ferramenta nos deu toda a base para o desenvolvimento dos 
diversos critérios para a implementação dos requisitos de teste. 

	A ferramenta possui documentação disponível em recoder.sourceforge.net /doc/api/  tal documentação, que se trata de uma espécie de 
javaDoc, foi um dos empecilhos para o melhor andamento do projeto, dado que perdemos muito tempo tentando adivinhar qual seria o nome da classe que apresentava a funcionalidade que desejavamos. 
	
	O tutorial disponibilizado pelo Alexandre foi de vital importância, pois depois de não conseguirmos desenvolver o projeto 
com o codeCover já tinhamos tentado fazer o tutorial sugerido no site do recoder e não tinhamos conseguido terminá-lo. 

\section{Todas Decisões}
\mbox{}
	O critério de todas as decisões faz com que todas as decisões do programa sejam testadas em seus dois possíveis valores
(verdadeiro ou falso). Assim sendo, utilizamos a árvore de análise sintática gerada pelo recoder para pegar todos os métodos 
de todas as classes presentes na pasta passada como parâmetro. 
	
	Para cada método criamos uma nova árvore sintática local que analiza apenas aquele método.  A árvore sintática do método
é totalmente percorrida e se encontramos um decisão adicionamos ela em uma lista. Isso é feito através das classes do proprio recoder
(If.class, while.class, for.class, do.class). 
	
	Depois de termos todas as decisões do método, o programa duplica essas decisões e as diferencia apenas pelo atributo valor presente na classe
Decisão, uma fica com valor verdadeiro e outra com valor falso. 

\newpage

\section{Todas Condições}
\mbox{}

	O critério de todas as condições faz exatamente o que o de todas as decisões faz com mais alguns passos. Dentre esses passos a mais um 
possui uma grande diferença, o objeto instanciado que guarda as informações de uma decisão não é da classe Decisão.java e sim da classe 
DecisaoAux.java. A diferença principal entre as duas é que a DecisãoAux possui uma lista de condições, isso foi feito apenas para que o arquivo 
XML gerado fosse mais claro. Nessa abordagem cada decisão possui uma lista de condições. Para cada condição devemos setar o valor como falso
e na condição semelhante o valor verdadeiro. Assim cada decisão(DecisaoAux) possui 2n condições no arquivo xml (supondo uma decisão com n condições) dessas, n possuem valor falso e as outras n, com as mesmas informações das primeiras n, possuem valor verdadeiro. 

\section{MCDC}
\mbox{}

	O critério MCDC que implementamos executa os mesmos passos do critério sobre todas as decisões. Após possuirmos todas as decisões
devidamente preenchidas em uma estrutura de árvore onde todos os nós são decisões com exceção das folhas que são condições, criamos uma tabela verdade para aquela decisão e filtramos os testes independentes. Apenas os testes independentes são colocados no arquivo XML. Infelizmente não nos sobrou tempo para o desenvolvimento do método que eliminaria as máscaras. 

\section{XML}
\mbox{}

	Os arquivos XML foram criados a partir da biblioteca xstream.jar. Essa biblioteca converte um objeto qualquer em um arquivoXML utilizando como 
tags os nomes de seus atributos. Para isso criamos 3 classes, cada uma represetando um dos 3 critérios a serem analisados. As infomações foram obtidas a partir de toda a modelagem executada pelo nosso programa, que tenta simular um package que contém classes java

\section{Como Usar?}
 \mbox{}

Antes de usar o protótipo, recomendamos baixar e instalar o eclipse (versão para Java). Importe o projeto, Atividade4, para o eclipse e instale as bibliotecas asm-all-3.3.jar, backport-util-concurrent-3.1.jar, bsh-1.2b2.jar, junit.jar, retrotranslator\_mod\_for\_recoder.jar, retrotranslator\_mod\_src.zip, recoder.jar, retrotranslator-runtime-1.2.7-transformed.jar e xstream-1.3.1.jar, que se encontram dentro da pasta lib, no path de execução do seu projeto. Por fim, instale as bibliotecas netx.jar e plugin.jar, que também se encontram na pasta lib, no diretório de sua jvm (no linux /usr/lib/jvm/java-<versão da vm>/jre/lib/).
Para rodar o protótipo deve-se passar por parâmetro o caminho absoluto da pasta que contém as classes java a serem analisadas. A saída é
composta por 3 arquivos XML, cada um correspondente a dos critérios pedidos no enunciado. 

\section{Testes}
\mbox{}

	Em anexo no zip que será entregue no paca estão as três classes que utilizei para testar o programa e os três arquivos XML gedados.

\section{Conclusões}
\mbox{}

\begin{itemize}
	
	\item Pontos Positivos: 
		\begin{itemize}
			\item Estruturas em Árvore
			\item Aprender como o recoder funciona
			\item Projeto em java e utilização do github
		\end{itemize}
	\item Pontos Negativos:
		\begin{itemize}
			\item CodeCover
			\item Fases Anteriores do projeto. 
			\item Modelagem do problema
			\item Não termos feito na semana do break
			\item Sumiço de um dos membros do grupo. 
		\end{itemize}
\end{itemize}

\end{document}
